\documentclass[a4paper,11pt]{article}
\usepackage{listings}
\usepackage{CJKutf8}
\usepackage{minted}
\usepackage{xcolor}      %代码着色宏包
\usepackage{CJK}         %显示中文宏包
\lstset{
    basicstyle=\tt,
    %行号
    numbers=left,
    rulesepcolor=\color{red!20!green!20!blue!20},
    escapeinside=``,
    xleftmargin=2em,xrightmargin=0em, aboveskip=1em,
    %背景框
    framexleftmargin=2mm,
    frame=shadowbox,
    %样式
    keywordstyle=\color{blue}\bfseries,
    identifierstyle=\bf,
    numberstyle=\color[RGB]{0,192,192},
    commentstyle=\it\color[RGB]{96,96,96},
    stringstyle=\rmfamily\slshape\color[RGB]{128,0,0},
    }
% define the title
\author{langman}
\title{ACM-template}

\begin{document}
\maketitle
\section{head.set}

\begin{lstlisting}[language = C]
  #include<iostream>
  #include<cstdio>
  #include<cstring>
  #include<string>
  #include<algorithm>
  #include<queue>
  #include<stack>
  #include<vector>
  #include<cmath>
  #include<set>
  #include<cstdlib>
  #include<functional>
  #include<climits>
  #include<cctype>
  #include<iomanip>
  using namespace std;
  typedef long long ll;
  #define INF 0x3f3f3f3f
  const int mod = 1e9+7 ;
  #define clr(a,x) memset(a,x,sizeof(a))
  #define cle(a,n) for(int i=1;i<=n;i++)
  a.clear();
  const double eps = 1e-6;
  int main()
  {
  freopen("in.txt","r",stdin);
  freopen("out.txt","w",stdout);

  return 0;
  }

\end{lstlisting}
\section{DP}
\subsection{LIS/LDS}
\begin{CJK}{UTF8}{gkai}
最大上升子序列,下降,严格上,严格降.\\
nlogn的复杂度,调用库里面的函数\\
\end{CJK}
\begin{lstlisting}[language = C]

LIS(LDS)
template<class Cmp>
int LIS (Cmp cmp)(nlogn)
{
    static int m, end[N];
    m = 0;
    for (int i=0;i<n;i++)
    {
      int pos=lower_bound(end,end+m,a[i],cmp)-end;
      end[pos]=a[i],m+=pos==m;
    }
    return m;
}
    cout << LIS(less<int>()) << endl;
    cout << LIS(less_equal<int>()) << endl;
    cout << LIS(greater<int>()) << endl;
    cout << LIS(greater_equal<int>()) << endl;
\end{lstlisting}
\subsection{dp in bag}
\begin{CJK}{UTF8}{gkai}
背包的题目,是比较基础的dp类型的题\\
01和完全背包相对来说比较简单,n方 的复杂度\\
部分背包的话相对来说,因该是把部分背包换成01背包,降低复杂度\\
\\\\
他的状态点在于
当背包容量为x时,他的最佳状态
然后找出容量是x的时候能从
哪几个子状态转移过来。
难点在于:\\
1:背包的构造\\
2:背包状态转移方程的寻找\\
3:方向是从前到后,还是从后到前\\
4:dp维数的确定\\\\\\\\
\end{CJK}

\begin{lstlisting}[language = C]
// 0 1
for(int i = 0;i<num;i++)
{
  for(int j = v;j>=money[i];j++)
   {
     dp[j] = max(dp[j],dp[j-money[i]]+value[i]);
   }
}

// part bag (better make it into 0 1 bag)

//full bag
for(int i = 0;i<num;i++)
{
  for(int j = v;j>=money[i];j--)
  {
    dp[j] = max(dp[j],dp[j-money[i]]+value[i]);
  }
}
\end{lstlisting}
\subsection{dp in tree}











\end{document}
