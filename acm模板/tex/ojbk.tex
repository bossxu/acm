\documentclass[a4paper,11pt]{article}
\usepackage{listings}
\usepackage{CJKutf8}
\usepackage{minted}
\usepackage{picinpar,graphicx}
\usepackage{CJK}

%中文断行
\begin{document}
\begin{CJK}{UTF8}{gbsn}

拉格朗日插值法\\
这个是对于一个n次函数,我用n+1个点去确定的一种方法,相当于对于直线来说,我是两个点去确定这个玩意,但是现在我们是通过n+1个点去确定这个曲线
$$ L_n(x) = \sum_{i = 1}^n(\prod_{j=1,j \neq i}\frac{x-x_j}{x-x_i})*y_i$$
牛顿插值法\\
\inputminted{c++}{../scoure/math/lang.cpp}
这怎么说了,这个的好处在于我们不断对于一个函数进行加点的时候,不会导致之前的重新计算,好像也没什么用处
$$N_n(x) = f(x_0) + f(x_0,x_1)(x-x_0)+  .... +f(x_0,x_1,....,x_n)(x-x_0)(x-x_1)....(x - x_n)$$
$$ f(x_0,x_1...x_n) = \sum_{j = 0}^n(\frac{f_j}{\prod_{i =0,i \neq j}^n(x_j-x_i)} )$$
\end{CJK}
\end{document}
