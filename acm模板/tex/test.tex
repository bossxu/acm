\documentclass[a4paper,11pt]{article}
\usepackage{listings}
\usepackage{CJKutf8}
\usepackage{minted}
\usepackage{CJK}
\author{langman}
\title{acm-template}

\begin{document}
\maketitle
\begin{CJK}{UTF8}{gbsn}
\section{头文件}

\inputminted{c++}{../scoure/head.cpp}
\newpage
\section{图论}
\subsection{并查集}
\inputminted{c++}{../scoure/Graph_theory/bingchick.cpp}
\newpage
\subsection{最短路}
两种算法 但是要注意dijkstra无法处理负边的情况
\subsubsection{dijkstra}
需要注意的在于 可以更优化 我没写了 而且需要注意重边的情况
\inputminted{c++}{../scoure/Graph_theory/dijkstra.cpp}
\subsubsection{spfa}
需要注意的是怎么建边 双向边?
\inputminted{c++}{../scoure/Graph_theory/spfa.cpp}
\subsubsection{Flody}
这个就不写了,一个小dp
\newpage
\subsection{最小生成树}
这是个什么玩意呢 图里面是吧,找到n-1条边使得生成一颗树,然后他的边权之和最小
\inputminted{c++}{../scoure/Graph_theory/prime.cpp}



\end{CJK}



\end{document}
