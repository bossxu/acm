\documentclass[a4paper,11pt]{article}
\usepackage{listings}
\usepackage{CJKutf8}
\usepackage{minted}
\usepackage{picinpar,graphicx}
\usepackage{CJK}

\author{langman}
\title{ACM-template}
\usepackage{lastpage}%获得总页数
\usepackage{fancyhdr}
\pagestyle{fancy}
%以下命令中L--左侧 R--右侧 C--中间 O--奇数页 E--偶数页
\fancyhead[LO,RE]{SHU--langman}%奇数页左侧,偶数页右侧显示页眉
\fancyfoot[CO,RE]{practice making surprise}%奇数页中间,偶数页右侧页脚为空
\fancyfoot[LO,CE]{}%奇数页左侧,偶数页中间页脚为空
\fancyfoot[RO,LE]{\thepage\ of
\pageref{LastPage}}%奇数页右侧,偶数页左侧显示 当前页 of 总页数
\renewcommand{\headrulewidth}{0.4pt}%改为0pt即可去掉页眉下面的横线
\renewcommand{\footrulewidth}{0.4pt}
\begin{document}
\maketitle
\vspace{3cm}
\begin{CJK}{UTF8}{gbsn}
    \begin{figure}[!htb]
      \begin{center}
        \includegraphics[width=0.80\linewidth]{../scoure/tupian.jpg}
        \caption{stay hungry stay foolish}
      \end{center}
    \end{figure}
\newpage
\tableofcontents
\newpage
\section{头文件}
\inputminted{c++}{../scoure/head.cpp}
\newpage
\section{图论}
\subsection{二分图}
判断是否二分图
\inputminted{c++}{../scoure/Graph_theory/pan2fen.cpp}
求出最大匹配数
\inputminted{c++}{../scoure/Graph_theory/2fentu.cpp}
\subsection{并查集}
\inputminted{c++}{../scoure/Graph_theory/bingchick.cpp}
\newpage
\subsection{最短路}
两种算法 但是要注意dijkstra无法处理负边的情况
\subsubsection{dijkstra}
需要注意的在于 可以更优化 我没写了 而且需要注意重边的情况
\inputminted{c++}{../scoure/Graph_theory/dijkstra.cpp}
\subsubsection{spfa}
需要注意的是怎么建边 双向边?
\inputminted{c++}{../scoure/Graph_theory/spfa.cpp}
\subsubsection{Flody}
这个就不写了,一个小dp
\subsection{最小生成树}
这是个什么玩意呢 图里面是吧,找到n-1条边使得生成一颗树,然后他的边权之和最小
\inputminted{c++}{../scoure/Graph_theory/prime.cpp}
\subsection{最大流}
\subsubsection{Dinic}
板子先存着,坑定用的着
\inputminted{c++}{../scoure/Graph_theory/dinic.cpp}
\newpage
\section{字符串}
\subsubsection{kmp}
\paragraph{适用点}
这个主要用在,一个是:他的那个周期函数的运用.一个是那个单模板串,多个匹配串的形式.
最主要的运用就是他的那个失配函数的运用.这里就随便弄一个板子过来了,为了打的快一点.
\inputminted{c++}{../scoure/date/kmp.cpp}
\subsubsection{字典树}
这个是一个比较高级的东西,一般这个玩意和前缀有点关系.
\inputminted{c++}{../scoure/date/Trie.cpp}
\subsubsection{ac自动机}
呦呦呦这个就高级了,他是基于那两个东西,字典树和kmp所衍生出来的一个算法.
\inputminted{c++}{../scoure/date/aczidongji.cpp}
\section{常用数据结构}
\subsection{STL}
\inputminted{c++}{../scoure/date/duilie.cpp}
\subsection{树状数组}
适用于的方面是,单点更新,多次查询。
\inputminted{c++}{../scoure/date/treevector.cpp}
\subsection{前缀和}
适用于多次区间更新,一次查询,代码就不搞上去了。
\subsection{线段树}
现在的我对于这方面还不够熟悉,先留个板子
\inputminted{c++}{../scoure/date/treexian.cpp}
\subsection{高精度}
我个人习惯使用Java。就是怎么说呢,感觉要学会用自动补全以及几个常用的加减乘除
加add减sub乘mul除div模mod就很ok,然后那个大叔据都是从字符串转过来的,要熟悉一下字符串的一些常用函数.
\inputminted{java}{../scoure/date/bigjing.java}
\section{数学方面}
\subsection{三个特别的数}
\subsubsection{Fib 数列}
$$f(x) = f(x-1)+f(x-2)$$
$$f(0) = 0,f(1) = 1$$
\subsubsection{卡特兰 数}
$$\sum_{i=1}^n f_i*f_{n-i}=f_n$$
$$h(n) = C_{2n}^n - C_{2n-1}^{n}$$
注意它这个数字来自于什么情况。
\subsubsection{斯大林公式}
$$\sqrt{2*PI*n} * (\frac{n}{e})^n = n!$$
\subsection{数论}
第一个自然是最基础的欧几里得算法,欧几里得算法的用处有很多,求最大公倍数,解方程,很多。在后面的过程会把一些常见的板子列出来,一般来说这些板子
都已经经过验证,但是不好说对吧。简单题我们可以通过
一些模板直接得出答案,但是怎么说,这些对于难题
估计只能算工具,重要的是如何转换。
\subsubsection{素数}
素数筛法
\inputminted{c++}{../scoure/math/shai.cpp}
\subsubsection{梅森素数}
一个知识点吧,m是一个正整数,且$2^m-1$为素数,那么m一定为素数。\\
如果m是一个素数,$M_p = 2^p-1$是梅森数\\
如果p是一个素数,并且$M_p = 2^p-1$也是素数,那么称$M_p$为梅森素数\\
对梅森素数的判定是一个算法:\\
Lucas-Lehmer:
$r_k \equiv r_{k-1} -2$\quad mod$M_p$\quad $r_1 = 4$
当且仅当 $r_{p-1} \equiv 0$\quad mod$M_p$
\subsubsection{miller-robin快速判素数}
\inputminted{c++}{../scoure/math/miller_robin.cpp}
\subsubsection{欧几里得}
\inputminted{c++}{../scoure/math/GCD.cpp}
然后是基于这个定理得出的一个定理,中国剩余定理
\inputminted{c++}{../scoure/math/chineseshengyu.cpp}
\subsubsection{乘法逆元}
思想是通过扩展欧几里得来得出,如果缘分到了,那么
还能用费马小定理来解
\inputminted{c++}{../scoure/math/niyuan.cpp}
\subsubsection{欧拉函数}
p为素数时$\phi(p) = p-1 $\\
a与n互质的时候 $a^{\phi(n)}\equiv 1$ mod $ n$\\
m,n互质$\phi(mn) = \phi(m) * \phi(n)$\\

这个东西好呀,他求的是比n小的,并且和n互质的数的个数
\inputminted{c++}{../scoure/math/oula.cpp}
更多的来说我觉得这个东西是一个工具,他对解一些题有很重要的作用,起到一个工具的作用
我目前学的比较浅,对他的优化作用没有很深的了解。
\subsubsection{莫比乌斯函数}
$F_n = \sum_{d|n}^n f_d$
$f_n = \sum_{d|n} u(d)*f(\frac{n}{d})$\\
$F_n = \sum_{n|d} f_d$
$f_n = \sum_{n|d} u(\frac{d}{n})*f(d)$\\
和欧拉函数一样很重要的一个函数他的定义我就不说了,毕竟我latex学的还不好,公式的
插入对我来说用处不大。
\inputminted{c++}{../scoure/math/mobius.cpp}
\newpage
\subsection{博弈}
\subsubsection{主要的解题思想}
官方说的是通过必败点和必胜点来判定
先通过必败点来推,直接来看必胜点,把问题抽象成图 把状态抽象成点,必败点就是先手必败点,然后通过必败点能走到的搞成必胜点,如过有一个状态没有走过 而且他后面的路都是必胜点那么他就是必败点。感觉就像dp一样,记忆化搜索。
当然题目不可能出的那么简单的。
不过根据雄爷定理,万事不离期宗,掌握基本,扩展自己去发掘。
\subsubsection{题型}
巴什博弈\\
这个是最简单的博弈,就是一堆东西,每个人自己能拿1-n件,谁最后一个拿完谁赢,这个是最简单的,不记录。
\\威佐夫博弈\\
有两堆各若干个物品,两个人轮流从某一堆或同时从两堆中取同样多的物品,规定每次至少取一个,多者不限,最后取光者得胜。
这个的解题思路在于通过前面的那个np问题来解决,用局势来思考这些问题,前几个局势在于(0,0),(1,2),(3,5),(4,7).....然后一些大佬就总结出了一些牛逼的结论$( a_k,b_k),a_k=\frac{k*(\sqrt{5}+1)}{2} , b_k=a_k+k$人才。
\\Fibonacci\\
有一堆个数为n的石子,游戏双方轮流取石子,满足:
\\(1)先手不能在第一次把所有的石子取完;
\\(2)之后每次可以取的石子数介于1到对手刚取的石子数的2倍之间(包含1和对手刚取的石子数的2倍)。 约定取走最后一个石子的人为赢家。
结论是 当n为Fibonacci数时,先手必败
\\尼姆博弈\\
有三堆各若干个物品,两个人轮流从某一堆取任意多的物品,规定每次至少取一个,多者不限,最后取光者得胜。
这个博弈有点意思 他的必败点的局势在于$(a,b,c) a \land {b \land c} = 0$
\subsubsection{SG函数}
这个在看之前感觉很高级但是啊,好像也就是一个dp的过程,通过一个必败点,看成起点然后,那个方法看成通向下一个起点的路,然后找所有能直接到这个必败点的必胜点。好像也就那么回事。好像能解决的都是小数字题这是一个板子,f里面存的是方法,多堆问题可以转化成异或来解决。
\inputminted{c++}{../scoure/math/boyi.cpp}
\subsubsection{解题策略}
* 1  :相信自己的第一感觉\\
* 2 :博弈都会和一些特别的数搭边 , 所以第一件事坑定是分析局势然后找找看是不是有特别的意义,像什么 卡特兰数,fib数列 ,幂次方,异或的值是否为0;\\
* 3  : 不挂怎么说,记得打表。\\
\subsection{组合数学}
\subsubsection{求组合数}
第一个是求组合数,方法很多不去列举,注意的是一般来说,组合数都是需要去
模一个数,所以他的分母在计算的时候是需要去求逆元的
\inputminted{c++}{../scoure/math/zuhe.cpp}
\subsubsection{polay定理}
设G={p1,p2,…,pt}是X={a1,a2,…,an}上一个置换群,用m种颜色对X中的元素进行涂色,那么不同的涂色方案数为
$$\frac{1}{G}\sum_{k=1}^t m^{Cyc(p_k)}$$
$Cyc(p_k)$是置换$p_k$的循环节个数
\subsubsection{Pell方程}
$$x^2-dy^2=1$$
当d不为平方数时,有无穷多的解。接下来是通项的推导
$$ x_{n+1} = x_1*x_n+d*y_1*y_n,y_{n+1} = x_1*y_n + y_1*x_n $$
暴力求$x_1,y_1$然后矩阵快速幂
\subsubsection{lucas定理}
当组合数的基数过大的时候进行这些操作但是注意,我们的操作也是要求那个模数为素数,且
模数要小的情况下,素数的情况我们可以用扩展lucas定理来解决。一个工具,一个数论上的分支。
\inputminted{c++}{../scoure/math/lucas.cpp}
\subsection{线代}
\subsubsection{fft优化多项式乘法法}
这个是一个工具,他的作用是加速多项式得乘法。这个点得难处还是在于多项式的构建,其实也不是
非常复杂得东西,当然他的原理,不敢去碰。关键词:多项式乘法,并且复杂度为1e5左右。
\inputminted{c++}{../scoure/math/fft.cpp}
\subsubsection{矩阵快速幂}
这类方法,很多是用在递推关系式的时候,像什么fib数列什么的。
\inputminted{c++}{../scoure/math/ju_quick.cpp}
\subsubsection{矩阵方面知识}
就是用高斯消元法去解决一些问题,像什么秩和方阵的值。
\subsection{计算几何}
\subsubsection{几何基本知识}
\emph{矢量}\\
矢量的乘积有很多的作用,注意定义。
适用点在于:1:面积 2:位置\\
\emph{跨立实验与判断两线段是否相交}\\
线段$P_1P_2,Q_1Q_2$,相交的条件为\\
$P_1Q_1$x$P_1P_2$  *  $P_1P_2$x$P_1Q_2>=0$\\
$Q_1P_1$x$Q_1Q_2$  *  $Q_1Q_2$x$Q_1P_2>=0$\\
\emph{pick定理}\\
线段上的是整数点的数的个数 求gcd;\\
PICK定理 设以整数点为顶点的多边形的面积为S,多边形内部的整数点数为N,多边形边界上的整数点数为L,则 S=L/2 + N-1

\subsubsection{判断点是否在多边形中}
\inputminted{c++}{../scoure/math/jihe.cpp}
为1的时候,则在内部。2,应该是边上。
\subsubsection{凸包问题}
\inputminted{c++}{../scoure/math/tubao.cpp}
\section{状态转移 dp}
dp的定义:
1:记忆化搜索;2:状态转移
所以我们的解决方案总是跟着这个来走,从定义出发。
难点集中于两个方面,状态式的确定和状态转移方程的确定
\subsection{背包}
背包的问题主要以下几种:
01背包,部分背包,完全背包;[相对来说比较简单] ,分组背包[个人感觉较难]
背包难在如何, 确定维数,确定背包的容量是什么以及背包的价值是什么,还有背包的dp关系转移式。
\subsection{一些常见的dp}
\subsubsection{LIS 最长上升子序列}
\inputminted{c++}{../scoure/dp/longerxuelie.cpp}
\subsection{树形dp}
关键点在于找状态点间的关系,他一般只有三个关系,父亲节点
,儿子节点,还有兄弟节点,去找他们之间的关系,所以一般是两遍dfs
找父亲与儿子的关系,找儿子与父亲的关系。
\subsection{数位dp}
这个dp的精髓在于记忆化搜索,也就是在最高位不是被限定的情况下进行记录,这样的话省掉很多多余的步骤。
所有的出发点都处于这个目的。
\subsection{状压dp}
这个dp的精髓在于状态转移,不过能压缩的情况也是很限定的。
像什么每个点的状态在于都是能用两个状态来描述,且这些点不多,但是组合的方式很多。
一些状压dp经常用的上的公式。
\inputminted{c++}{../scoure/dp/zhuangtai.cpp}
\subsection{the end}
\end{CJK}

\end{document}
