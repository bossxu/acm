\documentclass[a4paper,11pt]{article}
\usepackage{listings}
\usepackage{CJKutf8}
\usepackage{minted}
\usepackage{CJK}
\author{langman}
\title{acm-template}

\begin{document}
\maketitle
\begin{CJK}{UTF8}{gbsn}
\tableofcontents
\newpage
\section{头文件}
\inputminted{c++}{../scoure/head.cpp}
\newpage
\section{图论}
\subsection{并查集}
\inputminted{c++}{../scoure/Graph_theory/bingchick.cpp}
\newpage
\subsection{最短路}
两种算法 但是要注意dijkstra无法处理负边的情况
\subsubsection{dijkstra}
需要注意的在于 可以更优化 我没写了 而且需要注意重边的情况
\inputminted{c++}{../scoure/Graph_theory/dijkstra.cpp}
\subsubsection{spfa}
需要注意的是怎么建边 双向边?
\inputminted{c++}{../scoure/Graph_theory/spfa.cpp}
\subsubsection{Flody}
这个就不写了,一个小dp
\subsection{最小生成树}
这是个什么玩意呢 图里面是吧,找到n-1条边使得生成一颗树,然后他的边权之和最小
\inputminted{c++}{../scoure/Graph_theory/prime.cpp}
\subsection{最大流}
\subsubsection{Dinic}
板子先存着,坑定用的着
\inputminted{c++}{../scoure/Graph_theory/dinic.cpp}
\newpage
\section{数学方面}
\subsection{三个特别的数}
\subsubsection{Fib 数列}
$$f(x) = f(x-1)+f(x-2)$$
$$f(0) = 0,f(1) = 1$$
\subsubsection{卡特兰 数}
$$\sum_{i=1}^n f_i*f_{n-i}=f_n$$
$$h(n) = C_{2n}^n - C_{2n-1}^{n}$$
注意它这个数字来自于什么情况。
\subsubsection{斯大林公式}
$$ \sqrt{2*PI*n} * (\frac{n}{e})^n = n!$$
\subsection{数论}
第一个自然是最基础的欧几里得算法,欧几里得算法的用处有很多,求最大公倍数,解方程,很多。在后面的过程会把一些常见的板子列出来,一般来说这些板子
都已经经过验证,但是不好说对吧。简单题我们可以通过
一些模板直接得出答案,但是怎么说,这些对于难题
估计只能算工具,重要的是如何转换。
\subsubsection{欧几里得}
\inputminted{c++}{../scoure/math/GCD.cpp}
然后是基于这个定理得出的一个定理,中国剩余定理
\inputminted{c++}{../scoure/math/chineseshengyu.cpp}
\subsubsection{乘法逆元}
思想是通过扩展欧几里得来得出,如果缘分到了,那么
还能用费马小定理来解
\inputminted{c++}{../scoure/math/niyuan.cpp}
\subsubsection{欧拉函数}
这个东西好呀,他求的是比n小的,并且和n互质的数的个数
\inputminted{c++}{../scoure/math/oula.cpp}
更多的来说我觉得这个东西是一个工具,他对解一些题有很重要的作用,起到一个工具的作用
我目前学的比较浅,对他的优化作用没有很深的了解。
\subsubsection{莫比乌斯函数}
$$ F_n = \sum_{i=1}^n f_i $$
$$ f_i = \sum_{d|n} u(d)*f(\frac{d}{n}) $$
和欧拉函数一样很重要的一个函数他的定义我就不说了,毕竟我latex学的还不好,公式的
插入对我来说用处不大。
\inputminted{c++}{../scoure/math/mobius.cpp}
\newpage
\subsection{组合数学}
\subsubsection{求组合数}
第一个是求组合数,方法很多不去列举,注意的是一般来说,组合数都是需要去
模一个数,所以他的分母在计算的时候是需要去求逆元的
\inputminted{c++}{../scoure/math/zuhe.cpp}
\subsubsection{lucas定理}
当组合数的基数过大的时候进行这些操作但是注意,我们的操作也是要求那个模数为素数,且
模数要小的情况下,素数的情况我们可以用扩展lucas定理来解决。一个工具,一个数论上的分支。
\inputminted{c++}{../scoure/math/lucas.cpp}
\subsection{线代}
\subsubsection{矩阵快速幂}
这类方法,很多是用在递推关系式的时候,像什么fib数列什么的。
\inputminted{c++}{../scoure/math/ju_quick.cpp}
\subsubsection{矩阵方面知识}
就是用高斯消元法去解决一些问题,像什么秩和方阵的值。
\subsection{计算几何}
\subsubsection{几何基本知识}
\emph{矢量}\\
矢量的乘积有很多的作用,注意定义。
适用点在于:1:面积 2:位置\\
\emph{跨立实验与判断两线段是否相交}\\
线段$P_1P_2,Q_1Q_2$,相交的条件为\\
$P_1Q_1$x$P_1P_2$  *  $P_1P_2$x$P_1Q_2>=0$\\
$Q_1P_1$x$Q_1Q_2$  *  $Q_1Q_2$x$Q_1P_2>=0$\\
\emph{pick定理}\\
线段上的是整数点的数的个数 求gcd;\\
PICK定理 设以整数点为顶点的多边形的面积为S,多边形内部的整数点数为N,多边形边界上的整数点数为L,则 S=L/2 + N-1
\\
\subsubsection{判断点是否在多边形中}



\section{状态转移 dp}
dp的定义:
1:记忆化搜索;2:状态转移
所以我们的解决方案总是跟着这个来走,从定义出发。
难点集中于两个方面,状态式的确定和状态转移方程的确定
\subsection{背包}
背包的问题主要以下几种:
01背包,部分背包,完全背包;[相对来说比较简单] ,分组背包[个人感觉较难]
背包难在如何, 确定维数,确定背包的容量是什么以及背包的价值是什么,还有背包的dp关系转移式。
\subsection{树形dp}
关键点在于找状态点间的关系,他一般只有三个关系,父亲节点
,儿子节点,还有兄弟节点,去找他们之间的关系,所以一般是两遍dfs
找父亲与儿子的关系,找儿子与父亲的关系。
\subsection{数位dp}
这个dp的精髓在于记忆化搜索,也就是在最高位不是被限定的情况下进行记录,这样的话省掉很多多余的步骤。
所有的出发点都处于这个目的。
\subsection{状压dp}
这个dp的精髓在于状态转移,不过能压缩的情况也是很限定的。
像什么每个点的状态在于都是能用两个状态来描述,且这些点不多,但是组合的方式很多。
一些状压dp经常用的上的公式。
\inputminted{c++}{../scoure/dp/zhuangtai.cpp}
\subsection{一些常见的dp}
\subsubsection{LIS 最长上升子序列}
\inputminted{c++}{../scoure/dp/longerxuelie.cpp}

\section{The end}


\end{CJK}



\end{document}
