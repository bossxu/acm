\documentclass[a4paper,11pt]{article}
\usepackage{listings}
\usepackage{CJKutf8}
\usepackage{minted}
\usepackage{CJK}
\usepackage{picinpar,graphicx}
\author{langman}
\title{acm-template}

\usepackage{lastpage}%获得总页数
\usepackage{fancyhdr}
\pagestyle{fancy}
%以下命令中L--左侧 R--右侧 C--中间 O--奇数页 E--偶数页
\fancyhead[LO,RE]{SHU--langman}%奇数页左侧,偶数页右侧显示页眉
\fancyfoot[CO,RE]{practice making surprise}%奇数页中间,偶数页右侧页脚为空
\fancyfoot[LO,CE]{}%奇数页左侧,偶数页中间页脚为空
\fancyfoot[RO,LE]{\thepage\ of
\pageref{LastPage}}%奇数页右侧,偶数页左侧显示 当前页 of 总页数
\renewcommand{\headrulewidth}{0.4pt}%改为0pt即可去掉页眉下面的横线
\renewcommand{\footrulewidth}{0.4pt}%改为0pt即可去掉页脚上面的横线
\begin{document}
BaoBao is a big fan of the game and likes Sayori the most, so he decides to write a poem to please Sayori. A poem of $m$ words $s_1, s_2, \dots, s_m$ is nothing more than a sequence of $m$ strings, and the happiness of Sayori after reading the poem is calculated by the formula
where $H$ is the happiness and $f(s_i)$ is Sayori's preference to the word $s_i$.

Given a list of $n$ words and Sayori's preference to each word, please help BaoBao select $m$ words from the list and finish the poem with these $m$ words to maximize the happiness of Sayori.

Please note that each word can be used at most once!

Input

There are multiple test cases. The first line of input contains an integer $T$ (about 100), indicating the number of test cases. For each test case:

The first line contains two integers $n$ and $m$ ($1 \le m \le n \le 100$), indicating the number of words and the length of the poem.

For the following $n$ lines, the $i$-th line contains a string consisting of lowercased English letters $w_i$ ($1 \le |w_i| \le 15$) and an integer $f(w_i)$ ($-10^9 \le f(w_i) \le 10^9$), indicating the $i$-th word and Sayori's preference to this word. It's guaranteed that $w_i \ne w_j$ for all $i \ne j$.

Output

For each test case output one line containing an integer $H$ and $m$ strings $s_1, s_2, \dots, s_m$ separated by one space, indicating the maximum possible happiness and the corresponding poem. If there are multiple poems which can achieve the maximum happiness, print the lexicographically smallest one.

Please, DO NOT output extra spaces at the end of each line, or your answer may be considered incorrect!

A sequence of $m$ strings $a_1, a_2, \dots, a_m$ is lexicographically smaller than another sequence of $m$ strings $b_1, b_2, \dots, b_m$, if there exists a $k$ ($1 \le k \le m$) such that $a_i = b_i$ for all $1 \le i < k$ and $a_k$ is lexicographically smaller than $b_k$.

A string $s_1 = a_1a_2\dots a_x$ is lexicographically smaller than another string $s_2 = b_1b_2\dots b_y$, if there exists a $k$ ($1 \le k \le \min(x, y)$) such that $a_i = b_i$ for all $1 \le i < k$ and $a_k < b_k$, or $a_i = b_i$ for all $1 \le i \le \min(x, y)$ and $x < y$.




\newpage
adada
\end{document}
